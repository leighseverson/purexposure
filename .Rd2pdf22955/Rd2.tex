\documentclass[a4paper]{book}
\usepackage[times,inconsolata,hyper]{Rd}
\usepackage{makeidx}
\usepackage[utf8]{inputenc} % @SET ENCODING@
% \usepackage{graphicx} % @USE GRAPHICX@
\makeindex{}
\begin{document}
\chapter*{}
\begin{center}
{\textbf{\huge Package `purexposure'}}
\par\bigskip{\large \today}
\end{center}
\begin{description}
\raggedright{}
\inputencoding{utf8}
\item[Title]\AsIs{Pull and Calculate Exposure to CA Pesticide Use Registry Records}
\item[Version]\AsIs{0.0.0.9000}
\item[Description]\AsIs{Pulls and cleans California Pesticide Use Report (PUR) datasets
data to visualize and calculate exposure to active ingredients present
in applied pesticides.}
\item[Depends]\AsIs{R (>= 3.4.1)}
\item[License]\AsIs{GPL-2}
\item[Encoding]\AsIs{UTF-8}
\item[LazyData]\AsIs{TRUE}
\item[Imports]\AsIs{dplyr (>= 0.7.3),
tidyr (>= 0.7.1),
purrr (>= 0.2.3),
stringr (>= 1.2.0),
lubridate (>= 1.6.0),
rgdal (>= 1.2.8),
sp (>= 1.2.5),
broom (>= 0.4.2),
ggplot2 (>= 2.2),
ggmap (>= 2.6.1),
maps (>= 3.2.0),
maptools (>= 0.9.2),
geosphere (>= 1.5.5),
rlang (>= 0.1.2.9000),
rgeos (>= 0.3.23),
magrittr (>= 1.5),
utils (>= 3.4.1),
colormap (>= 0.1.9000),
grDevices (<= 3.4.1),
methods (>= 3.4.1),
zoo (>= 1.8.0)}
\item[RoxygenNote]\AsIs{6.0.1}
\item[Suggests]\AsIs{testthat,
knitr,
rmarkdown}
\item[VignetteBuilder]\AsIs{knitr}
\end{description}
\Rdcontents{\R{} topics documented:}
\inputencoding{utf8}
\HeaderA{calculate\_exposure}{Calculate exposure to active ingredients present in applied pesticides.}{calculate.Rul.exposure}
%
\begin{Description}\relax
For a particular location, buffer radius, date range, and active ingredient
or class of active ingredients, \code{calculate\_exposure} calculates an
estimate of exposure in kg of active ingredient per m\textasciicircum{}2.
\end{Description}
%
\begin{Usage}
\begin{verbatim}
calculate_exposure(clean_pur_df, location, radius, time_period = NULL,
  start_date = NULL, end_date = NULL, chemicals = "all",
  aerial_ground = FALSE, verbose = TRUE)
\end{verbatim}
\end{Usage}
%
\begin{Arguments}
\begin{ldescription}
\item[\code{clean\_pur\_df}] A data frame returned by \code{pull\_clean\_pur} that
includes data for the county of your location (before running
\code{pull\_clean\_pur}, you can use the \code{find\_location\_county}
function to figure this out), the time period, and the active ingredients
or chemical classes for which you want to calculate exposure.

\item[\code{location}] A length-one character string. Either a California address
including street name, city, state, and zip code, or a pair of coordinates
in the form "longitude, latitude".

\item[\code{radius}] A numeric value greater than zero that gives the radius in meters
defining the buffer around your location in which you would like to
calculate exposure. For reference, the length and width of a PLS section is
about 1,609 meters (1 mile). That of a township could range from about
9,656 to 11,265 meters (6-7 miles).

\item[\code{time\_period}] Optional. A character string giving a time period over which you
would like to calculate exposure. For example, if you enter "6 months" for
\code{time\_period}, \code{calculate\_exposure} will calculate exposure for
every six month period starting from the earliest date present in the
\code{clean\_pur\_df} data frame. Start and end dates can be optionally specified
with the \code{start\_date} and \code{end\_date} arguments. Alternatively, to
calculate exposure over only one time period, you can leave this argument
NULL and specify start and end dates.

\item[\code{start\_date}] Optional. "yyyy-mm-dd" specifying the start date for
exposure estimation. This date should be present in the \code{clean\_pur\_df}
data frame.

\item[\code{end\_date}] Optional. "yyyy-mm-dd" specifying the end date for exposure
estimation. This date should be present in the \code{clean\_pur\_df}
data frame.

\item[\code{chemicals}] Either "all" or "chemical\_class". The default is "all", which
will calculate exposure to the summed active ingredients present in the
\code{clean\_pur\_df} data frame. Enter "chemical\_class" to calculate
exposure to each of the chemical classes present in the \code{chemical\_class}
column of your \code{clean\_pur\_df} data frame.

\item[\code{aerial\_ground}] TRUE / FALSE for whether you would like to
incorporate aerial/ground application into exposure calculations. If
\code{aerial\_ground = TRUE}, there should be an \code{aerial\_ground}
column in the input \code{clean\_pur\_df} data frame. There will be a value of
exposure calculated for each chemical ("all" or by chemical class) and for
each method of application: aerial or ground. The default is FALSE.

\item[\code{verbose}] TRUE / FALSE for whether you would like a message to print out
while the function is running. The default is \code{TRUE}.
\end{ldescription}
\end{Arguments}
%
\begin{Value}
A list with five elements:
\begin{description}

\item[exposure] A data frame with 7 columns: \code{exposure},
the estimate of exposure in kg/m\textasciicircum{}2, \code{chemicals}, (either "all",
indicating that all active ingredients present in the \code{clean\_pur\_df}
were summed or the chemical class(es) specified in the \code{clean\_pur\_df}
data frame), \code{start\_date}, \code{end\_date}, \code{aerial\_ground},
which can take values of "A" = aerial, "G" = ground, and "O" = others, (if
the \code{aerial\_ground} argument is \code{FALSE}, \code{aerial\_ground}
will be \code{NA} in the \code{exposure} data frame), \code{location}, and
\code{radius}, the radius in meters for the buffer extending from the
location
\item[meta\_data] A data frame with 12 columns and at least one row for
every section or township intersected by the specified buffer extending
from the given location. Columns include \code{pls}, giving either the
Public Land Survey (PLS) section (9 characters long) or township (7
characters long), \code{chemicals}, \code{percent}, the percent that the
PLS unit is overlapped by the buffer, \code{kg}, the total amount of kg
applied for the specified chemicals and date range in that section or
township, \code{kg\_intersection}, the amount of kilograms applied
multiplied by the percent of overlap, \code{start\_date} and \code{end\_date},
\code{aerial\_ground}, which can take values of "A" (aerial), "G" (ground),
or "O" (other), and will be \code{NA} if exposure calculations did not
take aerial/ground application into account, \code{none\_recorded}, logical
for whether any pesticide application was recorded for the specified section
or township, date range, and chemicals, \code{location}, and \code{radius}
\item[buffer\_plot] A data frame with 24 columns. Contains spatial plotting
data for the buffer and overlapping sections or townships. You can use the
\code{df\_plot} function to quickly plot and get a rough idea of the
area for which exposure was calculated, before moving on to other map\_*
or plot\_* functions.
\item[county\_plot] A ggplot2 plot showing the location of your specified
buffer in the context of the county. Depending on if your \code{clean\_pur\_df}
data frame was summed by section or township, the county will be shown
with the relevant PLS units.
\item[clean\_pur\_df] The data frame supplied to the \code{clean\_pur\_df}
argument, filtered to the county and date range for which exposure
was calcualted.

\end{description}

\end{Value}
%
\begin{Section}{Note}

\begin{itemize}

\item If the \code{time\_period}, \code{start\_date}, and \code{end\_date}
arguments are all left as NULL (their defaults), then exposure will
be estiamted across the entire date range of the \code{clean\_pur\_df}
data frame.
\item If you pulled PUR data from \code{pull\_clean\_pur} specifying
\code{sum\_application = TRUE} and \code{unit = "township"}, then
exposure will be calculated based on townships. Using the
\code{df\_plot} function to plot the returned \code{buffer\_plot}
list element to take a look at the \code{county\_plot} plot element
could be helpful to see the difference between calculating exposure
based on sections or townships for a certain buffer radius.
\item This function takes advantage of the Google Maps Geocoding API, and
is limited by the standard usage limit of 2,500 free requests per
day and 50 requests per second.
\url{https://developers.google.com/maps/documentation/geocoding/usage-limits}

\end{itemize}

\end{Section}
%
\begin{Examples}
\begin{ExampleCode}
## Not run:
clean_pur <- pull_clean_pur(2000, counties = "10")
exposure_list <- calculate_exposure(clean_pur,
                                    location = "13883 Lassen Ave, Helm, CA 93627",
                                    radius = 3000)
exposure_list$county_plot

# specify time intervals
exp_list2 <- calculate_exposure(clean_pur,
                                location = "13883 Lassen Ave, Helm, CA 93627",
                                radius = 3000,
                                time_period = "4 months")
exp_list2$exposure

# calculate exposure by township
clean_pur2 <- pull_clean_pur(1995, counties = "san bernardino",
                             sum_application = TRUE, unit = "township")
exp_list3 <- calculate_exposure(clean_pur2,
                                location = "-116.45, 34.96",
                                radius = 5000)
df_plot(exp_list3$buffer_plot)
exp_list3$county_plot

# calculate exposure by specified chemical classes
# this is an example of `none_recorded = TRUE`
chemical_class_df <- rbind(find_chemical_codes(2000, "methylene"),
                           find_chemical_codes(2000, "aldehyde")) %>%
   dplyr::rename(chemical_class = chemical)
clean_pur3 <- pull_clean_pur(1995, "fresno",
                             chemicals = chemical_class_df$chemname,
                             sum_application = TRUE,
                             sum = "chemical_class",
                             chemical_class = chemical_class_df)
exp_list4 <- calculate_exposure(clean_pur3,
                                location = "13883 Lassen Ave, Helm, CA 93627",
                                radius = 1500,
                                chemicals = "chemical_class")
exp_list4$meta_data

# incorporate aerial/ground application information
clean_pur4 <- pull_clean_pur(2000, "yolo")
exp_list5 <- calculate_exposure(clean_pur4,
                                location = "-121.9018, 38.7646",
                                radius = 2500,
                                aerial_ground = TRUE)
exp_list5$exposure

## End(Not run)
\end{ExampleCode}
\end{Examples}
\inputencoding{utf8}
\HeaderA{california\_shp}{California state SpatialPolygonsDataFrame object.}{california.Rul.shp}
\keyword{datasets}{california\_shp}
%
\begin{Description}\relax
A SpatialPolygonsDataFrame object for the outline of the state of California.
Downloaded and subset from the 2016 U.S. Census Cartographic Boundary
Shapefile for states.
\end{Description}
%
\begin{Usage}
\begin{verbatim}
california_shp
\end{verbatim}
\end{Usage}
%
\begin{Format}
A SpatialPolygonsDataFrame object at the 1:20,000,000 resolution level.
\end{Format}
%
\begin{Source}\relax
\url{https://www.census.gov/geo/maps-data/data/cbf/cbf_state.html}
\end{Source}
\inputencoding{utf8}
\HeaderA{chemical\_list}{California Pesticide Use Report chemical codes.}{chemical.Rul.list}
\keyword{datasets}{chemical\_list}
%
\begin{Description}\relax
A list of data frames (one for each year from 1990 through 2015) containing
California Department of Pesticide Regulation chemical codes and names used
to identify active ingredients in Pesticide Use Report data. "chemical.txt"
files for each year were pulled from the .zip files "pur1990.zip" through
"pur2015.zip" found here:
\url{ftp://transfer.cdpr.ca.gov/pub/outgoing/pur_archives}
\end{Description}
%
\begin{Usage}
\begin{verbatim}
chemical_list
\end{verbatim}
\end{Usage}
%
\begin{Format}
A list of 26 elements. Each element is a data frame with a variable
number of rows ranging from 3,579 to 3,934 and two columns:
\begin{description}

\item[chem\_code] An integer giving the chemical code. This uniquely
identifies a chemical within a year.
\item[chemname] A character vector giving common chemical name for each
active ingredient. These are usually listed on the pesticide product
label.

\end{description}
\end{Format}
%
\begin{Source}\relax
\url{ftp://transfer.cdpr.ca.gov/pub/outgoing/pur_archives}
\end{Source}
\inputencoding{utf8}
\HeaderA{county\_codes}{California Pesticide Use Report county codes.}{county.Rul.codes}
\keyword{datasets}{county\_codes}
%
\begin{Description}\relax
A data frame containing California county names and corresponding codes
used to identify counties in California Pesticide Use Reports. This file,
"county.txt", was pulled from the .zip file "pur2000.zip" found here:
\url{ftp://transfer.cdpr.ca.gov/pub/outgoing/pur_archives}
\end{Description}
%
\begin{Usage}
\begin{verbatim}
county_codes
\end{verbatim}
\end{Usage}
%
\begin{Format}
A data frame with 58 rows and two columns:
\begin{description}

\item[county\_name] A character vector giving California county names.
\item[county\_code] A character vector giving county codes (ranging from "01"
through "58") corresponding to each California county. Note: these codes
are unique to California PUR datasets; they do not correspond to FIPS
codes.

\end{description}
\end{Format}
%
\begin{Source}\relax
\url{ftp://transfer.cdpr.ca.gov/pub/outgoing/pur_archives}
\end{Source}
\inputencoding{utf8}
\HeaderA{df\_plot}{Plot data frame spatial objects.}{df.Rul.plot}
%
\begin{Description}\relax
\code{df\_plot} plots a data frame spatial object. (A
SpatialPolygonsDataFrame that has been "tidied" using the broom package.)
Meant to be analogous to the ease of using plot() to quickly view a
SpatialPolygonDataFrame object.
\end{Description}
%
\begin{Usage}
\begin{verbatim}
df_plot(df)
\end{verbatim}
\end{Usage}
%
\begin{Arguments}
\begin{ldescription}
\item[\code{df}] A data frame returned from the \code{spdf\_to\_df} function.
\end{ldescription}
\end{Arguments}
%
\begin{Value}
A ggplot2 plot of the county.
\end{Value}
%
\begin{Examples}
\begin{ExampleCode}
## Not run:
shp <- pull_spdf("san diego", "township")
df <- spdf_to_df(shp)
df_plot(df)

## End(Not run)
\end{ExampleCode}
\end{Examples}
\inputencoding{utf8}
\HeaderA{euc\_distance}{Calculate euclidean distance between two points.}{euc.Rul.distance}
%
\begin{Description}\relax
\code{euc\_distance} calculates the straight-line distance between
two points.
\end{Description}
%
\begin{Usage}
\begin{verbatim}
euc_distance(long, lat, origin_long, origin_lat)
\end{verbatim}
\end{Usage}
%
\begin{Arguments}
\begin{ldescription}
\item[\code{long}] Longitude (x) of second point

\item[\code{lat}] Latitude (y) of second point

\item[\code{origin\_long}] Longitude (x) of first point

\item[\code{origin\_lat}] Latitude (y) of first point
\end{ldescription}
\end{Arguments}
%
\begin{Details}\relax
This is a helper function for \code{calculate\_exposure}.
\end{Details}
%
\begin{Value}
A data frame with one row and three columns: \code{long} and
\code{lat} give the second point's coordinates, and \code{dist} gives the
euclidian distance from these coordinates from the origin.
\end{Value}
\inputencoding{utf8}
\HeaderA{find\_chemical\_codes}{Pull active ingredient chemical codes from PUR Chemical Lookup Tables.}{find.Rul.chemical.Rul.codes}
%
\begin{Description}\relax
For a vector of chemical names, \code{find\_chemical\_codes} returns
a data frame with corresponding chemical codes from the PUR Chemical Lookup
Table for a given year. This function uses pattern matching to return results.
As a starting place, or for more thorough classifications, see the CA
Department of Pesticide Regulation's Summary of Pesticide Use Report Data,
Indexed by Chemical (2008):
\url{http://www.cdpr.ca.gov/docs/pur/pur08rep/chmrpt08.pdf}
\end{Description}
%
\begin{Usage}
\begin{verbatim}
find_chemical_codes(year, chemicals = "all")
\end{verbatim}
\end{Usage}
%
\begin{Arguments}
\begin{ldescription}
\item[\code{year}] A four-digit numeric year in the range of 1990 to 2015. Indicates
the year in which you would like to match chemical codes.

\item[\code{chemicals}] A string or vector of strings giving search terms of
chemicals to match with active ingredients present in pesticides applied
in the given year. The default value is "all", which returns codes for all
active ingredients applied in a given year.
\end{ldescription}
\end{Arguments}
%
\begin{Value}
A data frame with three columns:
\begin{description}

\item[chem\_code] An integer value with chemical codes corresponding to
each active ingredient. \code{chem\_code} values are used to later filter
raw PUR datasets.
\item[chemname] A character string giving unique active ingredients
corresponding to each search term.
\item[chemical] A character string with search terms given in the
\code{chemicals} argument.

\end{description}

\end{Value}
%
\begin{Section}{Note}

The PUR Chemical Lookup Table for a year lists all active ingredients present
in applied pesticides across the state of California. Therefore, PUR data for
a particular county may not include records for active ingredients returned
by \code{find\_chemical\_codes} for the same year.
\end{Section}
%
\begin{Examples}
\begin{ExampleCode}
find_chemical_codes(2000, "methyl bromide")
find_chemical_codes(1995, c("ammonia", "benzene"))
\end{ExampleCode}
\end{Examples}
\inputencoding{utf8}
\HeaderA{find\_counties}{Find California county codes or names.}{find.Rul.counties}
%
\begin{Description}\relax
Given a vector of counties, \code{find\_counties} returns either PUR
county codes or names.
\end{Description}
%
\begin{Usage}
\begin{verbatim}
find_counties(counties, return = "codes")
\end{verbatim}
\end{Usage}
%
\begin{Arguments}
\begin{ldescription}
\item[\code{return}] Either "codes" to return county codes (the default) or "names"
to return county names.

\item[\code{county}] A vector of character strings giving either a county names or
two digit PUR county codes. Not case sensitive. California names and county
codes as they appear in PUR datasets can be found in the \code{county\_codes}
dataset available with this package.
\end{ldescription}
\end{Arguments}
%
\begin{Value}
If \code{return = "codes"}, a vector of two-character strings giving
the corresponding PUR county codes. If \code{return = "names"}, a vector
of county names.
\end{Value}
%
\begin{Examples}
\begin{ExampleCode}
find_counties(c("01", "03", "el dorado"))
find_counties(c("contra costa", "45"))

find_counties(c("01", "03", "el dorado"), return = "names")
find_counties(c("contra costa", "45"), return = "names")
find_counties("fresno")
\end{ExampleCode}
\end{Examples}
\inputencoding{utf8}
\HeaderA{find\_location\_county}{Find the county from an address or coordinate pair.}{find.Rul.location.Rul.county}
%
\begin{Description}\relax
Given a California address or longitude/latitude coordinates,
\code{find\_location\_county} returns the corresponding California county or
PUR code.
\end{Description}
%
\begin{Usage}
\begin{verbatim}
find_location_county(location, return = "name", latlon_out = NULL)
\end{verbatim}
\end{Usage}
%
\begin{Arguments}
\begin{ldescription}
\item[\code{location}] A length-one character string. Either a California address
including street name, city, state, and zip code, or a pair of coordinates
in the form "longitude, latitude".

\item[\code{return}] Either "name" to return county name (the default) or "code"
to return county code.

\item[\code{latlon\_out}] A numeric vector of two with longitude and latitude
values. If the \code{geocode} code has been run earlier and this output is
available, this saves a redundant request to the Google Maps API.
\end{ldescription}
\end{Arguments}
%
\begin{Value}
A character string giving the California county where the address or
coordinate pair given in \code{location} is located.
\end{Value}
%
\begin{Examples}
\begin{ExampleCode}
## Not run:
address <- "13883 Lassen Ave, Helm, CA 93627"
find_location_county(location = address)

long_lat <- c("-120.09789, 36.53379")
find_location_county(location = long_lat)

## End(Not run)
\end{ExampleCode}
\end{Examples}
\inputencoding{utf8}
\HeaderA{find\_product\_name}{Find Pesticide Product names and registration numbers from PUR Product Lookup Tables.}{find.Rul.product.Rul.name}
%
\begin{Description}\relax
For a year and vector of product search terms, \code{find\_product\_name} returns
a data frame with corresponding product registration numbers, \code{prodno},
indicator codes, and product names.
\end{Description}
%
\begin{Usage}
\begin{verbatim}
find_product_name(year, products = "all", download_progress = FALSE)
\end{verbatim}
\end{Usage}
%
\begin{Arguments}
\begin{ldescription}
\item[\code{year}] A four digit year in the range of 1990 to 2015.

\item[\code{products}] A character string or a vector of character strings with
pesticide product names that you would like to search for. Not case
sensitive. The default is "all", which will return all pesticide products
applied for a given year.

\item[\code{download\_progress}] TRUE / FALSE indicating whether you would like a
message and progress bar printed for the product table that is downloaded.
The default value is FALSE.
\end{ldescription}
\end{Arguments}
%
\begin{Value}
A data frame with six columns:
\begin{description}

\item[prodno] The CA registration number. Can be matched with the
\code{prodno} in a raw or cleaned PUR dataset.
\item[prodstat\_ind] Character. An indication of product registration status:
\begin{itemize}

\item A = Active
\item B = Inactive
\item C = Inactive, Not Renewed
\item D = Inactive, Voluntary Cancellation
\item E = Inactive, Cancellation
\item F = Inactive, Suspended
\item G = Inactive, Invalid Data
\item H = Active, Suspended
\end{itemize}

\item[product\_name] Character. The name of the product taken from the
registered product label. May have been modified by DPR's Registration Branch
to ensure uniqueness.
\item[signlwrd\_ind] Integer. The signal word printed on the front of the
product label:
\begin{itemize}

\item 1 = Danger (Poison)
\item 2 = Danger (Only)
\item 3 = Warning
\item 4 = Caution
\item 5 = None
\end{itemize}

\item[year] The year for which product table information was pulled.
\item[product] Product name search terms.

\end{description}

\end{Value}
%
\begin{Examples}
\begin{ExampleCode}
## Not run:
prod_df <- find_product_name(2000, "mosquito")
prod_df <- find_product_name(2010, c("insecticide", "rodenticide"))

## End(Not run)
\end{ExampleCode}
\end{Examples}
\inputencoding{utf8}
\HeaderA{gradient\_n\_pal2}{Include alpha option in scales::gradient\_n\_pal().}{gradient.Rul.n.Rul.pal2}
%
\begin{Description}\relax
This function adds an "alpha" argument from gradient\_n\_pal() from the scales
package.
\end{Description}
%
\begin{Usage}
\begin{verbatim}
gradient_n_pal2(colours, values = NULL, space = "Lab", alpha = NULL)
\end{verbatim}
\end{Usage}
\inputencoding{utf8}
\HeaderA{help\_calculate\_exposure}{Return exposure data for a single start and end date.}{help.Rul.calculate.Rul.exposure}
%
\begin{Description}\relax
For a single date range, \code{help\_calculate\_exposure} returns the
\code{exposure} and \code{meta\_data} data frames to be output in the
\code{calculate\_exposure} list as a nested data frame.
\end{Description}
%
\begin{Usage}
\begin{verbatim}
help_calculate_exposure(start_date, end_date, aerial_ground, chemicals,
  clean_pur_df, location, pls_percents, pur_filt, radius)
\end{verbatim}
\end{Usage}
%
\begin{Arguments}
\begin{ldescription}
\item[\code{start\_date}] Optional. "yyyy-mm-dd" specifying the start date for
exposure estimation. This date should be present in the \code{clean\_pur\_df}
data frame.

\item[\code{end\_date}] Optional. "yyyy-mm-dd" specifying the end date for exposure
estimation. This date should be present in the \code{clean\_pur\_df}
data frame.

\item[\code{aerial\_ground}] TRUE / FALSE for whether you would like to
incorporate aerial/ground application into exposure calculations. If
\code{aerial\_ground = TRUE}, there should be an \code{aerial\_ground}
column in the input \code{clean\_pur\_df} data frame. There will be a value of
exposure calculated for each chemical ("all" or by chemical class) and for
each method of application: aerial or ground. The default is FALSE.

\item[\code{chemicals}] Either "all" or "chemical\_class". The default is "all", which
will calculate exposure to the summed active ingredients present in the
\code{clean\_pur\_df} data frame. Enter "chemical\_class" to calculate
exposure to each of the chemical classes present in the \code{chemical\_class}
column of your \code{clean\_pur\_df} data frame.

\item[\code{clean\_pur\_df}] A data frame returned by \code{pull\_clean\_pur} that
includes data for the county of your location (before running
\code{pull\_clean\_pur}, you can use the \code{find\_location\_county}
function to figure this out), the time period, and the active ingredients
or chemical classes for which you want to calculate exposure.

\item[\code{location}] A length-one character string. Either a California address
including street name, city, state, and zip code, or a pair of coordinates
in the form "longitude, latitude".

\item[\code{pls\_percents}] A data frame

\item[\code{pur\_filt}] A data frame

\item[\code{radius}] A numeric value greater than zero that gives the radius in meters
defining the buffer around your location in which you would like to
calculate exposure. For reference, the length and width of a PLS section is
about 1,609 meters (1 mile). That of a township could range from about
9,656 to 11,265 meters (6-7 miles).
\end{ldescription}
\end{Arguments}
%
\begin{Details}\relax
This is a helper function for \code{calculate\_exposure}.
\end{Details}
%
\begin{Value}
A nested data frame with two columns: The \code{row\_out} column
contains the \code{exposure} data frame for the date range, and
\code{meta\_data} contains the \code{meta\_data} data frame for the date range.
\end{Value}
\inputencoding{utf8}
\HeaderA{help\_calc\_exp}{Return a single exposure value for each combination of conditions.}{help.Rul.calc.Rul.exp}
%
\begin{Description}\relax
\code{help\_calc\_exp} returns a data frame with exposure values (kg/m\textasciicircum{}2) by
chemicals (including \code{chemicals = "all"}) or by chemicals and aerial/ground
application.
\end{Description}
%
\begin{Usage}
\begin{verbatim}
help_calc_exp(exp, buffer_area, ...)
\end{verbatim}
\end{Usage}
%
\begin{Arguments}
\begin{ldescription}
\item[\code{exp}] A data frame

\item[\code{...}] Either \code{chemicals} or \code{chemicals, aerial\_ground}. Not
quoted
\end{ldescription}
\end{Arguments}
%
\begin{Details}\relax
This is a helper function for \code{help\_calculate\_exposure}.
\end{Details}
%
\begin{Value}
A data frame with exposure values in kg/m\textasciicircum{}2 at a location for each
relevant condition.
\end{Value}
\inputencoding{utf8}
\HeaderA{help\_categorize}{Categorize a continuous scale by percentile cutpoints.}{help.Rul.categorize}
%
\begin{Description}\relax
Given a data frame with a column with a continuous numeric variable,
\code{help\_categorize} returns the data frame with a new column categorizing
that continuous variable by percentile cutpoints.
\end{Description}
%
\begin{Usage}
\begin{verbatim}
help_categorize(section_data, buffer_or_county, start_date = NULL,
  end_date = NULL, aerial_ground = NULL, chemicals = NULL,
  clean_pur = NULL, s_t = NULL, percentile)
\end{verbatim}
\end{Usage}
%
\begin{Arguments}
\begin{ldescription}
\item[\code{section\_data}] A data frame with a continuous numeric variable named "kg"

\item[\code{buffer\_or\_county}] Should cutpoints be determined by "county" or "buffer"?

\item[\code{start\_date}] Optional. "yyyy-mm-dd" specifying the start date for
exposure estimation. This date should be present in the \code{clean\_pur\_df}
data frame.

\item[\code{end\_date}] Optional. "yyyy-mm-dd" specifying the end date for exposure
estimation. This date should be present in the \code{clean\_pur\_df}
data frame.

\item[\code{aerial\_ground}] TRUE / FALSE for whether you would like to
incorporate aerial/ground application into exposure calculations. If
\code{aerial\_ground = TRUE}, there should be an \code{aerial\_ground}
column in the input \code{clean\_pur\_df} data frame. There will be a value of
exposure calculated for each chemical ("all" or by chemical class) and for
each method of application: aerial or ground. The default is FALSE.

\item[\code{chemicals}] Either "all" or "chemical\_class". The default is "all", which
will calculate exposure to the summed active ingredients present in the
\code{clean\_pur\_df} data frame. Enter "chemical\_class" to calculate
exposure to each of the chemical classes present in the \code{chemical\_class}
column of your \code{clean\_pur\_df} data frame.

\item[\code{clean\_pur}] A \code{pull\_clean\_pur} data frame

\item[\code{s\_t}] "section" or "township"

\item[\code{percentile}] A numeric vector in (0, 1) specifying percentile cutpoints
if \code{color\_by = "percentile"}. The default is \code{c(0.25, 0.5, 0.75)},
which results in four categories: < 25th percentile, >= 25th to < 50th,
>= 50th to < 75th, and >= 75th.
\end{ldescription}
\end{Arguments}
%
\begin{Details}\relax
This is a helper function for \code{help\_map\_exp}.
\end{Details}
%
\begin{Value}
The input \code{section\_data} data frame with an additional column
named \code{category}.
\end{Value}
\inputencoding{utf8}
\HeaderA{help\_filter\_pls}{Find PLS units intersecting with a buffer.}{help.Rul.filter.Rul.pls}
%
\begin{Description}\relax
\code{help\_filter\_pls} filters a SpatialPolygonsDataFrame to include only PLS
units intersecting with a buffer, and filters the data frame returned from
\code{pull\_clean\_pur} to include only those sections or townships.
\end{Description}
%
\begin{Usage}
\begin{verbatim}
help_filter_pls(pls, pls_quote, which_pls, shp, buffer, df, clean_pur_df)
\end{verbatim}
\end{Usage}
%
\begin{Arguments}
\begin{ldescription}
\item[\code{pls}] Either \code{MTRS} (sections) or \code{MTR} (townships). Not quoted

\item[\code{pls\_quote}] Either \code{"MTRS"} or \code{"MTR"}

\item[\code{which\_pls}] A vector of character string PLS units

\item[\code{shp}] A county's shapefile

\item[\code{buffer}] A data frame with buffer coordinates

\item[\code{df}] A data frame

\item[\code{clean\_pur\_df}] A data frame returned by \code{pull\_clean\_pur} that
includes data for the county of your location (before running
\code{pull\_clean\_pur}, you can use the \code{find\_location\_county}
function to figure this out), the time period, and the active ingredients
or chemical classes for which you want to calculate exposure.
\end{ldescription}
\end{Arguments}
%
\begin{Details}\relax
This is a helper function for \code{calculate\_exposure}.
\end{Details}
%
\begin{Value}
A list with four elements:
\begin{description}

\item[pur\_filt] A cleaned PUR data frame filtered to PLS units intersecting
with a buffer.
\item[comb\_df\_filt] A spatial data frame with intersecting PLS units and
a buffer.
\item[pls\_intersections] A data frame with two columns: \code{pls} and
\code{percent}, the corresponding percent intersection with the buffer.
\item[pls\_int] A character vector with all PLS units intersecting with the
buffer.

\end{description}

\end{Value}
\inputencoding{utf8}
\HeaderA{help\_find\_chemical}{Pull a chemical code based on its name.}{help.Rul.find.Rul.chemical}
%
\begin{Description}\relax
This function uses grep to return \code{chemname} values that match an input
search term.
\end{Description}
%
\begin{Usage}
\begin{verbatim}
help_find_chemical(chemical, df)
\end{verbatim}
\end{Usage}
%
\begin{Arguments}
\begin{ldescription}
\item[\code{chemical}] A string giving search term of a
chemical to match with active ingredients present in pesticides applied
in the given year.

\item[\code{df}] A chemical table data frame.
\end{ldescription}
\end{Arguments}
%
\begin{Details}\relax
This is a helper function for \code{find\_chemical\_codes}.
\end{Details}
%
\begin{Value}
A data frame with three columns: \code{chem\_code}, \code{chemname},
and \code{chemical}.
\end{Value}
\inputencoding{utf8}
\HeaderA{help\_find\_code}{Find California county code or name}{help.Rul.find.Rul.code}
%
\begin{Description}\relax
Given a county, \code{help\_find\_code} returns either the PUR
county code or name.
\end{Description}
%
\begin{Usage}
\begin{verbatim}
help_find_code(county, return = "codes")
\end{verbatim}
\end{Usage}
%
\begin{Arguments}
\begin{ldescription}
\item[\code{county}] A character string giving either a county name or
two digit PUR county code. Not case sensitive. California names and county
codes as they appear in PUR datasets can be found in the \code{county\_codes}
dataset available with this package.

\item[\code{return}] Either "codes" to return county code (the default) or "names"
to return county name.
\end{ldescription}
\end{Arguments}
%
\begin{Details}\relax
This is a helper function for \code{find\_counties}.
\end{Details}
%
\begin{Value}
If \code{return = "codes"}, a two-character string giving
the corresponding PUR county codes. If \code{return = "names"}, a county
name.
\end{Value}
%
\begin{Examples}
\begin{ExampleCode}
help_find_code("01", return = "names")
help_find_code("contra costa", return = "codes")
\end{ExampleCode}
\end{Examples}
\inputencoding{utf8}
\HeaderA{help\_find\_product}{Pull a chemical code based on its name.}{help.Rul.find.Rul.product}
%
\begin{Description}\relax
This function uses grep to return \code{chemname} values that match an input
search term.
\end{Description}
%
\begin{Usage}
\begin{verbatim}
help_find_product(product, df)
\end{verbatim}
\end{Usage}
%
\begin{Arguments}
\begin{ldescription}
\item[\code{df}] A chemical table data frame.

\item[\code{chemical}] A string giving search term of a
chemical to match with active ingredients present in pesticides applied
in the given year.
\end{ldescription}
\end{Arguments}
%
\begin{Details}\relax
This is a helper function for \code{find\_chemical\_codes}.
\end{Details}
%
\begin{Value}
A data frame with three columns: \code{chem\_code}, \code{chemname},
and \code{chemical}.
\end{Value}
\inputencoding{utf8}
\HeaderA{help\_map\_exp}{Return a map for a given exposure estimate.}{help.Rul.map.Rul.exp}
%
\begin{Description}\relax
For a unique combination of time periods, chemicals, and application methods,
\code{help\_map\_exp} returns a plot showing amounts of pesticides applied for PLS
units intersecting with the buffer.
\end{Description}
%
\begin{Usage}
\begin{verbatim}
help_map_exp(start_date, end_date, chemicals, aerial_ground, none_recorded,
  data_pls, gradient, location_longitude, location_latitude, buffer_df, buffer2,
  buffer, buffer_or_county, alpha, clean_pur, pls_labels, pls_labels_size,
  percentile, color_by)
\end{verbatim}
\end{Usage}
%
\begin{Arguments}
\begin{ldescription}
\item[\code{start\_date}] Optional. "yyyy-mm-dd" specifying the start date for
exposure estimation. This date should be present in the \code{clean\_pur\_df}
data frame.

\item[\code{end\_date}] Optional. "yyyy-mm-dd" specifying the end date for exposure
estimation. This date should be present in the \code{clean\_pur\_df}
data frame.

\item[\code{chemicals}] Either "all" or "chemical\_class". The default is "all", which
will calculate exposure to the summed active ingredients present in the
\code{clean\_pur\_df} data frame. Enter "chemical\_class" to calculate
exposure to each of the chemical classes present in the \code{chemical\_class}
column of your \code{clean\_pur\_df} data frame.

\item[\code{aerial\_ground}] TRUE / FALSE for whether you would like to
incorporate aerial/ground application into exposure calculations. If
\code{aerial\_ground = TRUE}, there should be an \code{aerial\_ground}
column in the input \code{clean\_pur\_df} data frame. There will be a value of
exposure calculated for each chemical ("all" or by chemical class) and for
each method of application: aerial or ground. The default is FALSE.

\item[\code{data\_pls}] A data frame

\item[\code{gradient}] A character vector of hex color codes

\item[\code{location\_longitude}] A numeric longitude value

\item[\code{location\_latitude}] A numeric latitude value

\item[\code{buffer\_df}] A data frame

\item[\code{buffer2}] A data frame

\item[\code{buffer}] A gpc.poly object

\item[\code{buffer\_or\_county}] "buffer" or "county"

\item[\code{alpha}] A number in [0,1] specifying the transperency of fill colors.
Numbers closer to 0 will result in more transparency. The default is 0.7.

\item[\code{clean\_pur}] A \code{clean\_pur\_df} data frame

\item[\code{pls\_labels}] TRUE / FALSE for whether you would like sections or townships
to be labeled with their PLS ID. The default is \code{FALSE}.

\item[\code{pls\_labels\_size}] A number specifying the size of PLS labels. The default
is 4.

\item[\code{percentile}] A numeric vector in (0, 1) specifying percentile cutpoints
if \code{color\_by = "percentile"}. The default is \code{c(0.25, 0.5, 0.75)},
which results in four categories: < 25th percentile, >= 25th to < 50th,
>= 50th to < 75th, and >= 75th.

\item[\code{color\_by}] Either "amount" (the default) or "percentile". Specifies
whether you would like application amounts to be colored according to
amount, resulting in a gradient legend, or by the percentile that they fall
into for the given dataset and date range. You can specify percentile
cutpoints with the \code{percentile} argument.
\end{ldescription}
\end{Arguments}
%
\begin{Details}\relax
This is a helper function for \code{plot\_exposure}.
\end{Details}
%
\begin{Value}
A list with three elements:
\begin{description}

\item[plot] An exposure plot
\item[data] A data frame with information about each PLS unit
\item[cutoff\_values] A data frame with cutoff values returned if
\code{color\_by = "percentile"}

\end{description}

\end{Value}
\inputencoding{utf8}
\HeaderA{help\_read\_in\_counties}{Read in PUR county dataset for a year.}{help.Rul.read.Rul.in.Rul.counties}
%
\begin{Description}\relax
Once a year of PUR data has been downloaded, this function reads in a
selected county's dataset.
\end{Description}
%
\begin{Usage}
\begin{verbatim}
help_read_in_counties(code_or_file, type, year)
\end{verbatim}
\end{Usage}
%
\begin{Arguments}
\begin{ldescription}
\item[\code{code\_or\_file}] A PUR county code or a file name for a county's dataset.

\item[\code{type}] Either "codes" or "files", specifying the type of argument supplied
to \code{code\_or\_file}.
\end{ldescription}
\end{Arguments}
%
\begin{Details}\relax
This is a helper function for \code{pull\_pur\_file}.
\end{Details}
%
\begin{Value}
A data frame of raw PUR data for a single county and year.
\end{Value}
\inputencoding{utf8}
\HeaderA{help\_remove\_cols}{Remove columns with all missing values.}{help.Rul.remove.Rul.cols}
%
\begin{Description}\relax
Given a quoted column name and its data frame, \code{help\_remove\_cols} determines
if that column has all missing values or not.
\end{Description}
%
\begin{Usage}
\begin{verbatim}
help_remove_cols(col_quote, df)
\end{verbatim}
\end{Usage}
%
\begin{Arguments}
\begin{ldescription}
\item[\code{col\_quote}] A quoted column name

\item[\code{df}] A data frame
\end{ldescription}
\end{Arguments}
%
\begin{Details}\relax
This is a helper function for \code{help\_sum\_application}.
\end{Details}
%
\begin{Value}
A data frame with two columns: \code{col} gives the column name, and
\code{all\_missing} is a logical value.
\end{Value}
\inputencoding{utf8}
\HeaderA{help\_return\_exposure}{Return a data frame with exposure values and other related data.}{help.Rul.return.Rul.exposure}
%
\begin{Description}\relax
For a single date range, \code{help\_return\_exposure} returns a data frame with
exposure values calculated from \code{help\_calc\_exp} as well as other relevant
data. This one row data frame is combined with data frames for other date
ranges and then returned as the \code{exposure} element from a
\code{calculate\_exposure} list.
\end{Description}
%
\begin{Usage}
\begin{verbatim}
help_return_exposure(start_date, end_date, location, radius, exp, buffer_area,
  ...)
\end{verbatim}
\end{Usage}
%
\begin{Arguments}
\begin{ldescription}
\item[\code{start\_date}] Optional. "yyyy-mm-dd" specifying the start date for
exposure estimation. This date should be present in the \code{clean\_pur\_df}
data frame.

\item[\code{end\_date}] Optional. "yyyy-mm-dd" specifying the end date for exposure
estimation. This date should be present in the \code{clean\_pur\_df}
data frame.

\item[\code{location}] A length-one character string. Either a California address
including street name, city, state, and zip code, or a pair of coordinates
in the form "longitude, latitude".

\item[\code{radius}] A numeric value greater than zero that gives the radius in meters
defining the buffer around your location in which you would like to
calculate exposure. For reference, the length and width of a PLS section is
about 1,609 meters (1 mile). That of a township could range from about
9,656 to 11,265 meters (6-7 miles).

\item[\code{exp}] A data frame

\item[\code{...}] Either \code{chemicals} or \code{chemicals, aerial\_ground}. Not
quoted.
\end{ldescription}
\end{Arguments}
%
\begin{Details}\relax
This is a helper function for \code{help\_calculate\_exposure}.
\end{Details}
%
\begin{Value}
A data frame with one row and the columns found in the
\code{calculate\_exposure\$exposure} data frame.
\end{Value}
\inputencoding{utf8}
\HeaderA{help\_sum\_ai}{Find summed applied active ingredients.}{help.Rul.sum.Rul.ai}
%
\begin{Description}\relax
\code{help\_sum\_ai} finds the summed amount of applied active ingredients by
section or township, chemical class, and aerial/ground application.
\end{Description}
%
\begin{Usage}
\begin{verbatim}
help_sum_ai(pur_filt, start_date, end_date, ...)
\end{verbatim}
\end{Usage}
%
\begin{Arguments}
\begin{ldescription}
\item[\code{pur\_filt}] A data frame

\item[\code{start\_date}] Optional. "yyyy-mm-dd" specifying the start date for
exposure estimation. This date should be present in the \code{clean\_pur\_df}
data frame.

\item[\code{end\_date}] Optional. "yyyy-mm-dd" specifying the end date for exposure
estimation. This date should be present in the \code{clean\_pur\_df}
data frame.

\item[\code{...}] A list of variables to group by. Options include \code{section},
\code{township}, \code{chemical\_class}, and \code{aerial\_ground}. Not quoted
\end{ldescription}
\end{Arguments}
%
\begin{Details}\relax
This is a helper function for \code{help\_calculate\_exposure}.
\end{Details}
%
\begin{Value}
A data frame a \code{kg} column and one to three additional columns,
depending on the grouping variables.
\end{Value}
\inputencoding{utf8}
\HeaderA{help\_sum\_application}{Sum application by section, township, chemical, and method of application.}{help.Rul.sum.Rul.application}
%
\begin{Description}\relax
\code{help\_sum\_application} sums application of a PUR dataset by chemicals,
PLS unit, and aerial/ground application.
\end{Description}
%
\begin{Usage}
\begin{verbatim}
help_sum_application(df, sum, unit, aerial_ground, section_townships,
  chemical_class = NULL, ...)
\end{verbatim}
\end{Usage}
%
\begin{Arguments}
\begin{ldescription}
\item[\code{df}] A data frame from \code{pull\_clean\_pur} before summing has taken
place

\item[\code{sum}] A character string giving either "all" (the
default) or "chemical\_class". If \code{sum\_application = TRUE},
\code{sum} indicates whether you would like to sum across all active
ingredients, giving an estimation of the total pesticides applied in a
given section or township ("all"), or by a chemical class specified in a
data frame given in the argument \code{chemical\_class}.

\item[\code{unit}] A character string giving either "section" or "township".
Specifies whether applications of each active ingredient should be summed
by California section (the default) or by township. Only used if
\code{sum\_application} is \code{TRUE}.

\item[\code{aerial\_ground}] TRUE / FALSE indicating if you would like to
retain aerial/ground application data ("A" = aerial, "G" = ground, and
"O" = other.) The default is FALSE.

\item[\code{section\_townships}] A section\_townships data frame

\item[\code{chemical\_class}] A data frame with only three columns: \code{chem\_code},
\code{chemname}, and \code{chemical\_class}. \code{chem\_code} should have
integer values giving PUR chemical codes, and \code{chemname} should have
character strings with corresponding PUR chemical names (these can be
searched for using the \code{find\_chemical\_codes} function or with the
\code{chemical\_list} dataset included with this package). The
\code{chemical\_class} column should have character strings indicating the
chemical class corresponding to each \code{chem\_code}. The
\code{chemical\_class} for a group of active ingredients should be decided
upon by the user. Only used if \code{sum = "chemical\_class"}. See
the CDPR's Summary of PUR Data document here:
\url{http://www.cdpr.ca.gov/docs/pur/pur08rep/chmrpt08.pdf} for
comprehensive classifications of active ingredients.

\item[\code{...}] grouping variables
\end{ldescription}
\end{Arguments}
%
\begin{Details}\relax
This is a helper function for \code{pull\_clean\_pur}.
\end{Details}
%
\begin{Value}
A data frame. The number of columns is dependent on the grouping
variables supplied to the \code{...} argument.
\end{Value}
\inputencoding{utf8}
\HeaderA{help\_write\_md}{Return a \code{meta\_data} data frame.}{help.Rul.write.Rul.md}
%
\begin{Description}\relax
\code{help\_write\_md} returns a data frame to be output as the \code{meta\_data}
element in the list returned from \code{calculate\_exposure}.
\end{Description}
%
\begin{Usage}
\begin{verbatim}
help_write_md(clean_pur_df, pls_percents, pur_out, location, start_date,
  end_date, radius, buffer_area, mtrs_mtr, section_township)
\end{verbatim}
\end{Usage}
%
\begin{Arguments}
\begin{ldescription}
\item[\code{clean\_pur\_df}] A data frame returned by \code{pull\_clean\_pur} that
includes data for the county of your location (before running
\code{pull\_clean\_pur}, you can use the \code{find\_location\_county}
function to figure this out), the time period, and the active ingredients
or chemical classes for which you want to calculate exposure.

\item[\code{pls\_percents}] A data frame

\item[\code{pur\_out}] A data frame

\item[\code{location}] A length-one character string. Either a California address
including street name, city, state, and zip code, or a pair of coordinates
in the form "longitude, latitude".

\item[\code{start\_date}] Optional. "yyyy-mm-dd" specifying the start date for
exposure estimation. This date should be present in the \code{clean\_pur\_df}
data frame.

\item[\code{end\_date}] Optional. "yyyy-mm-dd" specifying the end date for exposure
estimation. This date should be present in the \code{clean\_pur\_df}
data frame.

\item[\code{radius}] A numeric value greater than zero that gives the radius in meters
defining the buffer around your location in which you would like to
calculate exposure. For reference, the length and width of a PLS section is
about 1,609 meters (1 mile). That of a township could range from about
9,656 to 11,265 meters (6-7 miles).

\item[\code{buffer\_area}] A numeric value

\item[\code{mtrs\_mtr}] Either \code{MTRS} or \code{MTR}. Not quoted

\item[\code{section\_township}] Either \code{section} or \code{township}. Not quoted
\end{ldescription}
\end{Arguments}
%
\begin{Details}\relax
This is a helper function for \code{help\_calculate\_exposure}.
\end{Details}
%
\begin{Value}
A data frame with the twelve columns in the
\code{calculate\_exposure\$meta\_data} data frame.
\end{Value}
\inputencoding{utf8}
\HeaderA{plot\_application\_timeseries}{Plot time series of active ingredients in applied pesticides.}{plot.Rul.application.Rul.timeseries}
%
\begin{Description}\relax
\code{plot\_application\_timeseries} returns a \code{ggplot2} time series plot
of pesticides present in a \code{pull\_clean\_pur} data frame. You can choose
whether to facet the time series by active ingredient (\code{chemname}) or by
\code{chemical\_class}.
\end{Description}
%
\begin{Usage}
\begin{verbatim}
plot_application_timeseries(clean_pur_df, facet = FALSE, y_axis = "fixed")
\end{verbatim}
\end{Usage}
%
\begin{Arguments}
\begin{ldescription}
\item[\code{clean\_pur\_df}] A data frame returned from \code{pull\_clean\_pur}.

\item[\code{facet}] TRUE / FALSE for whether you would like time series
plots to be faceted by unqiue \code{chemname} or \code{chemical\_class}
column values. If \code{facet = FALSE} (the default), all active ingredients
present in the dataset will be summed per day.

\item[\code{y\_axis}] A character string passed on to the \code{scales} argument of
\code{ggplot2::facet\_wrap} (\code{"fixed"}, \code{"free"}, \code{"free\_x"},
or \code{"free\_y"}). The default is \code{"fixed"}.
\end{ldescription}
\end{Arguments}
%
\begin{Value}
A \code{ggplot2} object.
\end{Value}
%
\begin{Examples}
\begin{ExampleCode}
## Not run:
pull_clean_pur(1990:1992, "fresno") %>%
    dplyr::filter(chemname %in% toupper(c("methyl bromide", "sulfur"))) %>%
    plot_application_timeseries(facet = TRUE)

pull_clean_pur(2000, "riverside") %>% plot_application_timeseries()

## End(Not run)
\end{ExampleCode}
\end{Examples}
\inputencoding{utf8}
\HeaderA{plot\_county\_application}{Plot pesticide application by county.}{plot.Rul.county.Rul.application}
%
\begin{Description}\relax
\code{plot\_county\_application} returns a plot of applied pesticides (either the
sum of all active ingredients present in the input \code{pull\_clean\_pur} data
frame, a specified chemical class, or a specified active ingredient). Application
is summed by section or township. PLS units can be shaded by amount or by
percentile.
\end{Description}
%
\begin{Usage}
\begin{verbatim}
plot_county_application(clean_pur_df, county = NULL, pls = NULL,
  color_by = "amount", percentile = c(0.25, 0.5, 0.75), start_date = NULL,
  end_date = NULL, chemicals = "all", fill_option = "viridis",
  crop = FALSE, alpha = 1)
\end{verbatim}
\end{Usage}
%
\begin{Arguments}
\begin{ldescription}
\item[\code{clean\_pur\_df}] A data frame returned by \code{pull\_clean\_pur}.

\item[\code{county}] Optional. If your \code{clean\_pur\_df} data frame contains data
for multiple counties, this argument specifies which county you would like
to plot application for. Either a PUR county name or county code. California
names and county codes as they appear in PUR datasets can be found in the
county\_codes dataset available with this package.

\item[\code{pls}] Optional. Either "section" or "township". If your
\code{clean\_pur\_df} data frame has both a \code{section} and
\code{township} column, the \code{pls} argument specifies which pls unit
you would like to plot application for. If you pulled data specifying
\code{unit = "township"}, application will be plotted by township.

\item[\code{color\_by}] Either "amount" (the default) or "percentile". Specifies
whether you would like application amounts to be colored according to
amount, resulting in a gradient legend, or by the percentile that they fall
into for the given dataset and date range. You can specify percentile
cutpoints with the \code{percentile} argument.

\item[\code{percentile}] A numeric vector in (0, 1) specifying percentile cutpoints
if \code{color\_by = "percentile"}. The default is \code{c(0.25, 0.5, 0.75)},
which results in four categories: < 25th percentile, >= 25th to < 50th,
>= 50th to < 75th, and >= 75th.

\item[\code{start\_date}] Optional. "yyyy-mm-dd" giving a starting date for the date
range that you would like to map application for. The default is to plot
application for the entire date range in your \code{clean\_pur\_df} data frame.

\item[\code{end\_date}] Optional. "yyyy-mm-dd" giving an ending date for the date
range that you would like to plot application for. The default is to plot
application for the entire date range in your \code{clean\_pur\_df} data frame.

\item[\code{chemicals}] Either "all" (the default) to plot summed active ingredents
present in your \code{clean\_pur\_df} data frame, a chemical class present in
the \code{chemical\_class} column of the \code{clean\_pur\_df} data frame, or
a specific active ingredient present in the \code{chemname} column of the
\code{clean\_pur\_df} data frame.

\item[\code{fill\_option}] A palette from the colormap package. The default is
"viridis". See colormap palette options by visiting
\url{https://bhaskarvk.github.io/colormap/} or by running
\code{colormap::colormaps}.

\item[\code{crop}] TRUE / FALSE for whether you would like your plot zoomed in on
sections or townships with recorded application data.

\item[\code{alpha}] A number in [0,1] specifying the transperency of fill colors.
Numbers closer to 0 will result in more transparency. The default is 1.
\end{ldescription}
\end{Arguments}
%
\begin{Value}
A list with three elements:
\begin{description}

\item[map] A plot of the county with application summed by section or
township and colored by amount or by percentile.
\item[data] A data frame with the plotted application data.
\item[cutoff\_values] A data frame with two columns: \code{percentile}
and \code{kg}, giving the cut points for each percentile in the
\code{clean\_pur\_df} for the specified chemicals. This element of the list
is not returned if \code{color\_by = "amount"}.

\end{description}

\end{Value}
%
\begin{Examples}
\begin{ExampleCode}
## Not run:
tulare_list <- pull_clean_pur(2010, "tulare") %>% plot_county_application()

# plot all active ingredients
pur_df <- pull_clean_pur(2000:2001, "fresno", verbose = F)
fresno_list <- plot_county_application(pur_df, color_by = "percentile",
                                      percentile = c(0.2, 0.4, 0.6, 0.8))
fresno_list$map
head(fresno_list$data)
fresno_list$cutoff_values

# map a specific active ingredient
fresno_list2 <- plot_county_application(pur_df, pls = "township",
                                       chemicals = "sulfur",
                                       fill_option = "plasma")
fresno_list2$map

# map a chemical class
chemical_class_df <- purrr::map2_dfr(2010, c("methidathion", "parathion",
                                             "naled", "malathion",
                                             "trichlorfon"),
                                     find_chemical_codes) %>%
     dplyr::mutate(chemical_class = "organophosphates") %>%
     dplyr::select(-chemical)
op_yuba <- pull_clean_pur(2010, "yuba",
                          chemicals = chemical_class_df$chemname,
                          verbose = F, sum_application = T,
                          sum = "chemical_class",
                          chemical_class = chemical_class_df) %>%
   plot_county_application()
op_yuba$map

## End(Not run)

\end{ExampleCode}
\end{Examples}
\inputencoding{utf8}
\HeaderA{plot\_county\_locations}{Plot a county's location in California.}{plot.Rul.county.Rul.locations}
%
\begin{Description}\relax
\code{plot\_county\_locations} returns one or multiple plots with county
locations in California given either a vector of county names or codes,
or a PUR data frame with a \code{county\_cd}, \code{county\_name}, or
\code{county\_code} column (A data frame returned from either
\code{pull\_pur\_file}, \code{pull\_raw\_pur}, or \code{pull\_clean\_pur}).
\end{Description}
%
\begin{Usage}
\begin{verbatim}
plot_county_locations(counties_or_df, one_plot = TRUE, fill_color = "red",
  alpha = 0.5)
\end{verbatim}
\end{Usage}
%
\begin{Arguments}
\begin{ldescription}
\item[\code{counties\_or\_df}] A character vector of county names or county codes.
You can use the \code{county\_codes} dataset included with this package to
check out PUR county names and codes. This argument can also be a data frame
with a \code{county\_cd}, \code{county\_name}, or \code{county\_code} column. If
you provide a data frame, a plot for every county with data in that dataset
will be output.

\item[\code{one\_plot}] TRUE / FALSE. If you provided multiple counties, whether you
would like county outlines plotted in the same plot (TRUE), or if you would
like separate plots returned in a list (FALSE). The default is TRUE

\item[\code{fill\_color}] A character string giving either a ggplot2 color or a
hex color code ("\#0000FF", for example). The default is "red".

\item[\code{alpha}] A number in [0,1] specifying the transparency of the fill
color. Numbers closer to 0 will result in more transparency. The default is
0.5.
\end{ldescription}
\end{Arguments}
%
\begin{Value}
A ggplot or a list of ggplots of Califnornia with shaded-in counties.
List element names correspond to county names.
\end{Value}
%
\begin{Examples}
\begin{ExampleCode}
## Not run:
plot_county_locations("fresno")

pur_df <- pull_clean_pur(1990, counties = c("01", "05", "12"), verbose = FALSE)
plot_county_locations(pur_df)

plot_list <- plot_county_locations(c("san bernardino", "ventura"), one_plot = TRUE)
names(plot_list)
plot_list[[1]]
plot_list[[2]]

## End(Not run)

\end{ExampleCode}
\end{Examples}
\inputencoding{utf8}
\HeaderA{plot\_exposure}{Plot exposure to applied pesticides at a location.}{plot.Rul.exposure}
%
\begin{Description}\relax
\code{plot\_exposure} returns a plot of pesticide application in the PLS units
intersected by a buffer for each combination of time period, applied active
ingredients, and applicaiton method relevant for the exposure values returned
from \code{calculate\_exposure}.
\end{Description}
%
\begin{Usage}
\begin{verbatim}
plot_exposure(exposure_list, color_by = "amount",
  buffer_or_county = "county", percentile = c(0.25, 0.5, 0.75),
  fill_option = "viridis", alpha = 0.7, pls_labels = FALSE,
  pls_labels_size = 4)
\end{verbatim}
\end{Usage}
%
\begin{Arguments}
\begin{ldescription}
\item[\code{exposure\_list}] A list returned from \code{calculate\_exposure}.

\item[\code{color\_by}] Either "amount" (the default) or "percentile". Specifies
whether you would like application amounts to be colored according to
amount, resulting in a gradient legend, or by the percentile that they fall
into for the given dataset and date range. You can specify percentile
cutpoints with the \code{percentile} argument.

\item[\code{buffer\_or\_county}] Either "county" (the default) or "buffer". Specifies
whether you would like colors to be scaled according to the limits
of application within the buffer, or in the county for the same time period,
chemicals, and method of application.

\item[\code{percentile}] A numeric vector in (0, 1) specifying percentile cutpoints
if \code{color\_by = "percentile"}. The default is \code{c(0.25, 0.5, 0.75)},
which results in four categories: < 25th percentile, >= 25th to < 50th,
>= 50th to < 75th, and >= 75th.

\item[\code{fill\_option}] A palette from the colormap package. The default is
"viridis". See colormap palette options by visiting
\url{https://bhaskarvk.github.io/colormap/} or by running
\code{colormap::colormaps}.

\item[\code{alpha}] A number in [0,1] specifying the transperency of fill colors.
Numbers closer to 0 will result in more transparency. The default is 0.7.

\item[\code{pls\_labels}] TRUE / FALSE for whether you would like sections or townships
to be labeled with their PLS ID. The default is \code{FALSE}.

\item[\code{pls\_labels\_size}] A number specifying the size of PLS labels. The default
is 4.
\end{ldescription}
\end{Arguments}
%
\begin{Value}
A list with the following elements:
\begin{description}

\item[maps] A list of plots. One plot for each exposure value returned in
the \code{exposure} element of the \code{calculate\_exposure} list.
\item[pls\_data] A list of data frames with 12 columns: \code{pls}, giving
the PLS ID, \code{percent}, the
buffer, \code{kg}, the amount of kg of pesticides applied in that PLS unit
for the relevant time period, chemicals, and application method,
\code{kg\_intersection}, \code{kg} multiplied by \code{percent} (this is the
value that is plotted), \code{start\_date}, \code{end\_date}, \code{chemicals},
\code{aerial\_ground}, which give the time period, chemicals, and application
method for each plot/exposure estiamte, \code{none\_recorded}, \code{location},
\code{radius} (m), and \code{area} (m\textasciicircum{}2).
\item[cutoff\_values] A list of data frames with two columns: \code{percentile} and
\code{kg} giving the cutoff values for each percentile. Only returned if
\code{color\_by = "percentile"}.

\end{description}

\end{Value}
%
\begin{Examples}
\begin{ExampleCode}
## Not run:
tulare_list <- pull_clean_pur(2010, "tulare") %>%
   calculate_exposure(location = "-119.3473, 36.2077",
                      radius = 3500) %>%
   plot_exposure()
names(tulare_list)
tulare_list$maps
tulare_list$pls_data
tulare_list$exposure
tulare_list$cutoff_values

# return one plot, pls_data data frame, exposure row, and cutoff_values
data frame for each exposure combination

dalton_list <- pull_clean_pur(2000, "modoc") %>%
    calculate_exposure(location = "-121.4182, 41.9370",
                       radius = 4000,
                       time_period = "6 months",
                       aerial_ground = TRUE) %>%
    plot_exposure(fill_option = "plasma")
do.call("rbind", dalton_list$exposure)
# one map for each exposure value (unique combination of chemicals,
dates, and aerial/ground application)
dalton_list$maps[[1]]
dalton_list$maps[[2]]
dalton_list$maps[[3]]
dalton_list$maps[[4]]
dalton_list$maps[[5]]
dalton_list$maps[[6]]

# exposure to a particular active ingredient
# plot amounts instead of percentile categories
chemical_df <- rbind(find_chemical_codes(2009, c("metam-sodium")) %>%
     dplyr::rename(chemical_class = chemical)

santa_maria <- pull_clean_pur(2008:2010, "santa barbara",
                              chemicals = chemical_df$chemname,
                              sum_application = TRUE,
                              sum = "chemical_class",
                              chemical_class = chemical_df) %>%
     calculate_exposure(location = "-119.6122, 34.90635",
                        radius = 3000,
                        time_period = "1 year",
                        chemicals = "chemical_class") %>%
     plot_exposure("amount")
do.call("rbind", santa_maria$exposure)
santa_maria$maps[[1]]
santa_maria$maps[[2]]
santa_maria$maps[[3]]

# scale colors based on buffer or county
turk <- pull_clean_pur(1996, "fresno") %>%
     dplyr::filter(chemname == "PHOSPHORIC ACID") %>%
     calculate_exposure(location = "-120.218404, 36.1806",
                        radius = 1500)

plot_exposure(turk, buffer_or_county = "county")$maps
plot_exposure(turk, buffer_or_county = "buffer")$maps

plot_exposure(turk, "amount", buffer_or_county = "county", pls_labels = TRUE)$maps
plot_exposure(turk, "amount", buffer_or_county = "buffer", pls_labels = TRUE)$maps

## End(Not run)
\end{ExampleCode}
\end{Examples}
\inputencoding{utf8}
\HeaderA{pull\_clean\_pur}{Pull cleaned PUR data by counties, years, and active ingredients.}{pull.Rul.clean.Rul.pur}
%
\begin{Description}\relax
\code{pull\_clean\_pur} returns a data frame of cleaned Pesticide Use Report data
filtered by counties, years, and active ingredients. Active ingredients
or chemical classes present in applied pesticides can be summed by either
Public Land Survey (PLS) section or township.
\end{Description}
%
\begin{Usage}
\begin{verbatim}
pull_clean_pur(years = "all", counties = "all", chemicals = "all",
  sum_application = FALSE, unit = "section", sum = "all",
  chemical_class = NULL, aerial_ground = TRUE, verbose = TRUE,
  download_progress = TRUE, raw_pur_df = NULL)
\end{verbatim}
\end{Usage}
%
\begin{Arguments}
\begin{ldescription}
\item[\code{years}] A four-digit numeric year or vector of years in the range of
1990 to 2015. Indicates the years for which you would like to pull PUR
datasets. \code{years == "all"} will pull data from 1990 through 2015.

\item[\code{counties}] A vector of character strings giving either a county name or
a two digit county code for each county. Not case sensitive. California names
and county codes as they appear in PUR datasets can be found in the
\code{county\_codes} dataset available with this package. For example, to
return data for Alameda county, enter either "alameda" or "01" for the
\code{counties} argument. \code{counties = "all"} will return data for
all 58 California counties.

\item[\code{chemicals}] A string or vector of strings giving search terms of
chemicals to match with active ingredients present in pesticides applied in
the given years. The default value is "all", which returns records for all
active ingredients applied in a given year. See the CDPR's Summary of PUR
Data document here:
\url{http://www.cdpr.ca.gov/docs/pur/pur08rep/chmrpt08.pdf} for
comprehensive classifications of active ingredients.

\item[\code{sum\_application}] TRUE / FALSE indicating if you would like to sum the
amounts of applied active ingredients by day, the geographic unit
given in \code{unit}, and by either active ingredients or chemical class
(indicated by \code{sum} and \code{chemical\_class}). The default value
is FALSE.

\item[\code{unit}] A character string giving either "section" or "township".
Specifies whether applications of each active ingredient should be summed
by California section (the default) or by township. Only used if
\code{sum\_application} is \code{TRUE}.

\item[\code{sum}] A character string giving either "all" (the
default) or "chemical\_class". If \code{sum\_application = TRUE},
\code{sum} indicates whether you would like to sum across all active
ingredients, giving an estimation of the total pesticides applied in a
given section or township ("all"), or by a chemical class specified in a
data frame given in the argument \code{chemical\_class}.

\item[\code{chemical\_class}] A data frame with only three columns: \code{chem\_code},
\code{chemname}, and \code{chemical\_class}. \code{chem\_code} should have
integer values giving PUR chemical codes, and \code{chemname} should have
character strings with corresponding PUR chemical names (these can be
searched for using the \code{find\_chemical\_codes} function or with the
\code{chemical\_list} dataset included with this package). The
\code{chemical\_class} column should have character strings indicating the
chemical class corresponding to each \code{chem\_code}. The
\code{chemical\_class} for a group of active ingredients should be decided
upon by the user. Only used if \code{sum = "chemical\_class"}. See
the CDPR's Summary of PUR Data document here:
\url{http://www.cdpr.ca.gov/docs/pur/pur08rep/chmrpt08.pdf} for
comprehensive classifications of active ingredients.

\item[\code{aerial\_ground}] TRUE / FALSE indicating if you would like to
retain aerial/ground application data ("A" = aerial, "G" = ground, and
"O" = other.) The default is FALSE.

\item[\code{verbose}] TRUE / FALSE indicating whether you would like a single message
printed indicating which counties and years you are pulling data for. The
default value is TRUE.

\item[\code{download\_progress}] TRUE / FALSE indicating whether you would like a
message and progress bar printed for each year of PUR data that is downloaded.
The default value is TRUE.

\item[\code{raw\_pur\_df}] A raw PUR data frame. Optional. If you've already downloaded
a raw PUR data frame using \code{pull\_raw\_pur}, this argument prevents
\code{pull\_clean\_pur} from downloading the same data again.
\end{ldescription}
\end{Arguments}
%
\begin{Value}
A data frame with 13 columns:
\begin{description}

\item[chem\_code] An integer value giving the PUR chemical code
for the active ingredient applied. Not included if
\code{sum\_application = TRUE} and \code{sum = "chemical\_class"}.
\item[chemname] A character string giving PUR chemical active
ingredient names. Unique values of \code{chemname} are matched with terms
provided in the \code{chemicals} argument. Not included
if \code{sum\_application = TRUE} and \code{sum = "chemical\_class"}.
\item[chemical\_class] If \code{sum\_application = TRUE} and
\code{sum = "chemical\_class"}, this column will give values of the
\code{chemical\_class} column in the input \code{chemical\_class} data frame.
If there are active ingredients pulled based on the
\code{chemicals} argument that are not present in the \code{chemical\_class}
data frame, these chemicals will be summed under the class "other".
\item[kg\_chm\_used] A numeric value giving the amount of the active
ingredient applied (kilograms).
\item[section] A string nine characters long indicating the section
of application. PLS sections are uniquely identified by a combination of
base line meridian (S, M, or H), township (01-48), township direction
(N or S), range (01-47), range direction (E or W) and section number
(01-36). This column is not included if
\code{sum\_application = TRUE} and \code{unit = "township"}.
\item[township] A string 7 characters long indicating the township
of application. PLS townships are uniquely identified by a combination of
base line meridian (S, M, or H), township (01-48), township direction
(N or S), range (01-47), and range direction (E or W).
\item[county\_name] A character string giving the county name where
application took place.
\item[county\_code] A string two characters long giving the PUR county
code where application took place.
\item[date] The date of application (yyyy-mm-dd).
\item[aerial\_ground] A character giving the application method.
"A" = aerial, "G" = ground, and "O" = other. Not included
if \code{aerial\_ground = FALSE}.
\item[use\_no] A character string identifing unique application of an
active ingredient across years. This value is a combination of the raw PUR
\code{use\_no} column and the year of application. Will have values of
\code{NA} if \code{sum\_appliction = TRUE}.
\item[outlier] A logical value indicating whether the
amount listed in \code{kg\_chm\_used} has been corrected large amounts
entered in error. The algorithm for identifying and replacing outliers
was developed based on methods used by Gunier et al. (2001). Please see
the package vignette for more detail regarding these methods. Will have
values of \code{NA} if \code{sum\_application = TRUE}.
\item[prodno] Integer. The California Registration Number for the applied
pesticide (will be repeated for different active ingredients present in
the product). You can match product registration numbers with product
names, which can be pulled using the \code{pull\_product\_table} function.
This column is not returned if \code{sum\_application = TRUE}.

\end{description}

\end{Value}
%
\begin{Section}{Note}

\begin{itemize}

\item The \code{chemical\_list} data frame for a particular year lists
active ingredients present in applied pesticides across the state of
California. Therefore, PUR data for a particular county may not include
records for active ingredients listed in the \code{chemical\_list} dataset
for the same year.
\item To pull raw PUR data, see the \code{pull\_raw\_pur} function.
For documentation of raw PUR data, see the Pesticide Use Report Data User
Guide \& Documentation document published by the California Department of
Pesticide Regulation. This file is saved as "cd\_doc.pdf" in
any "pur[year].zip" file between 1990 and 2015 found here:
\url{ftp://transfer.cdpr.ca.gov/pub/outgoing/pur_archives/}.
\item If this function returns an error, check your working directory.
You may want to change it back from a temporary directory.

\end{itemize}

\end{Section}
%
\begin{Examples}
\begin{ExampleCode}
## Not run:
df <- pull_clean_pur(download_progress = TRUE)
df2 <- pull_clean_pur(years = 2000:2010,
                      counties = c("01", "nevada", "riverside"),
                      chemicals = "methylene",
                      aerial_ground = TRUE)

# filter to particular products
prod_nos <- find_product_name(2003, "insecticide") %>%
    dplyr::select(prodno) %>%
    tibble_to_vector()

df3 <- pull_clean_pur(2003) %>%
    dplyr::filter(prodno %in% prod_nos)

# Sum application by active ingredients
df4 <- pull_clean_pur(years = 2000:2010,
                      counties = c("01", "nevada", "riverside"),
                      chemicals = "methylene",
                      unit = "township", sum_application = TRUE)

# Or by chemical classes
chemical_class_df <- rbind(find_chemical_codes(2000, "methylene"),
                           find_chemical_codes(2000, "aldehyde")) %>%
   dplyr::rename(chemical_class = chemical)

df5 <- pull_clean_pur(years = 1995,
                      counties = "fresno",
                      chemicals = chemical_class_df$chemname,
                      sum_application = TRUE,
                      sum = "chemical_class",
                      unit = "township",
                      chemical_class = chemical_class_df)

# clean an existing raw PUR dataset
placer_05 <- pull_raw_pur(2005, "placer")
df6 <- pull_clean_pur(raw_pur_df = placer_05)

## End(Not run)
\end{ExampleCode}
\end{Examples}
\inputencoding{utf8}
\HeaderA{pull\_product\_table}{Pull PUR Product Table.}{pull.Rul.product.Rul.table}
%
\begin{Description}\relax
This function pulls a California Department of Pesticide Regulation Product
Table for a particular year.
\end{Description}
%
\begin{Usage}
\begin{verbatim}
pull_product_table(year, download_progress = FALSE)
\end{verbatim}
\end{Usage}
%
\begin{Arguments}
\begin{ldescription}
\item[\code{year}] A four digit year in the range of 1990 to 2015.

\item[\code{download\_progress}] TRUE / FALSE indicating whether you would like a
message and progress bar printed for the product table that is downloaded.
The default value is FALSE.
\end{ldescription}
\end{Arguments}
%
\begin{Value}
A data frame with four columns:
\begin{description}

\item[prodno] Integer. The California Registration number for the pesticide
product. This corresponds to the \code{prodno} column in a raw or cleaned PUR
dataset returned from \code{pull\_raw\_pur} or \code{pull\_clean\_pur}.
\item[prodstat\_ind] Character. An indication of product registration status:
\begin{itemize}

\item A = Active
\item B = Inactive
\item C = Inactive, Not Renewed
\item D = Inactive, Voluntary Cancellation
\item E = Inactive, Cancellation
\item F = Inactive, Suspended
\item G = Inactive, Invalid Data
\item H = Active, Suspended
\end{itemize}

\item[product\_name] Character. The name of the product taken from the
registered product label. May have been modified by DPR's Registration Branch
to ensure uniqueness.
\item[signlwrd\_ind] Integer. The signal word printed on the front of the
product label:
\begin{itemize}

\item 1 = Danger (Poison)
\item 2 = Danger (Only)
\item 3 = Warning
\item 4 = Caution
\item 5 = None
\end{itemize}

\item[year] Integer. Four digit year indicating the year for which data was
pulled.

\end{description}

\end{Value}
%
\begin{Examples}
\begin{ExampleCode}
## Not run:
prod_95 <- pull_product_table(1995)

## End(Not run)
\end{ExampleCode}
\end{Examples}
\inputencoding{utf8}
\HeaderA{pull\_pur\_file}{Pull raw PUR file for a single year and a county or counties.}{pull.Rul.pur.Rul.file}
%
\begin{Description}\relax
\code{pull\_pur\_file} pulls the raw PUR dataset for a particular year and
saves datasets for specified counties in a data frame.
\end{Description}
%
\begin{Usage}
\begin{verbatim}
pull_pur_file(year, counties = "all", download_progress = TRUE)
\end{verbatim}
\end{Usage}
%
\begin{Arguments}
\begin{ldescription}
\item[\code{counties}] A character vector giving either county names or two digit
county codes. Not case sensitive. California names and county codes as they
appear in PUR datasets can be found in the county\_codes dataset available
with this package. For example, to return data for Alameda county, enter
either "alameda" or "01" for the county argument. \code{counties = "all"}
will pull data for all 58 California counties.

\item[\code{download\_progress}] TRUE / FALSE indicating whether you would like a
message and progress bar printed for each year of PUR data that is downloaded.
The default value is TRUE.
\end{ldescription}
\end{Arguments}
%
\begin{Value}
A data frame with 33 columns. Counties are indicated by
\code{county\_cd}; the year for which data was pulled is indicated by
\code{applic\_dt}.
\end{Value}
%
\begin{Section}{Note}

If this function returns an error (because the FTP site is down, for
example), check your working directory. You may want to change it back from a
temporary directory.

\begin{Examples}
\begin{ExampleCode}
## Not run:
raw\_file <- pull\_pur\_file(1999, c("40", "ventura", "yuba"))
raw\_file <- pull\_pur\_file(2015, "all")

## End(Not run)
\end{ExampleCode}
\end{Examples}
\end{Section}
\inputencoding{utf8}
\HeaderA{pull\_raw\_pur}{Pull raw PUR data by counties and years.}{pull.Rul.raw.Rul.pur}
%
\begin{Description}\relax
\code{pull\_raw\_pur} pulls a raw PUR dataset for a given year and vector of
California counties.
\end{Description}
%
\begin{Usage}
\begin{verbatim}
pull_raw_pur(years = "all", counties = "all", verbose = TRUE,
  download_progress = TRUE)
\end{verbatim}
\end{Usage}
%
\begin{Arguments}
\begin{ldescription}
\item[\code{years}] A four-digit numeric year or vector of years in the range of
1990 to 2015. Indicates the years for which you would like to pull PUR
datasets. \code{years == "all"} will pull data from 1990 through 2015.

\item[\code{counties}] A vector of character strings giving either a county name or
a two digit county code for each county. Not case sensitive. California names
and county codes as they appear in PUR datasets can be found in the
\code{county\_codes} dataset available with this package. For example, to
return data for Alameda county, enter either "alameda" or "01" for the
\code{counties} argument. \code{counties = "all"} will return data for
all 58 California counties.

\item[\code{verbose}] TRUE / FALSE indicating whether you would like a single message
printed indicating which counties and years you are pulling data for. The
default value is TRUE.

\item[\code{download\_progress}] TRUE / FALSE indicating whether you would like a
message and progress bar printed for each year of PUR data that is downloaded.
The default value is TRUE.
\end{ldescription}
\end{Arguments}
%
\begin{Value}
A data frame with 33 columns. Different years and counties for which
data was pulled are indicated by \code{applic\_dt} and \code{county\_cd},
respectively.
\end{Value}
%
\begin{Section}{Note}

\begin{itemize}

\item For documentation of raw PUR data, see the Pesticide Use
Report Data User Guide \& Documentation document published by the California
Department of Pesticide Regulation. This file is saved as "cd\_doc.pdf" in any
"pur[year].zip" file between 1990 and 2015 found here:
\url{ftp://transfer.cdpr.ca.gov/pub/outgoing/pur_archives/}.
\item If this function returns an error (because the FTP site is down, for
example), check your working directory. You may want to change it back from
a temporary directory.

\end{itemize}

\end{Section}
%
\begin{Examples}
\begin{ExampleCode}
## Not run:
df <- pull_raw_pur(download_progress = TRUE) # this will take a while to run
df2 <- pull_raw_pur(years = c(2000, 2010), counties = c("butte", "15", "01"))
df3 <- pull_raw_pur(years = 2015, counties = c("colusa"))

## End(Not run)
\end{ExampleCode}
\end{Examples}
\inputencoding{utf8}
\HeaderA{pull\_spdf}{Pull California county SpatialPolygonsDataFrame.}{pull.Rul.spdf}
%
\begin{Description}\relax
\code{pull\_spdf} pulls either the section or township-level
SpatialPolygonsDataFrame from a county's Geographic Information System (GIS)
shapefile.
\end{Description}
%
\begin{Usage}
\begin{verbatim}
pull_spdf(county, section_township = "section", download_progress = FALSE)
\end{verbatim}
\end{Usage}
%
\begin{Arguments}
\begin{ldescription}
\item[\code{county}] A vector of character strings giving either a county names or
two digit PUR county codes. Not case sensitive. California names and county
codes as they appear in PUR datasets can be found in the \code{county\_codes}
dataset available with this package.

\item[\code{section\_township}] Either "section" (the default) or "township".
Specifies whether you would like to pull a section- or township-level
SpatialPolygonsDataFrame.

\item[\code{download\_progress}] TRUE / FALSE indicating whether you would like a
message and progress bar printed for the shapefile that is downloaded.
The default value is FALSE
\end{ldescription}
\end{Arguments}
%
\begin{Value}
A SpatialPolygonsDataFrame object.
\end{Value}
%
\begin{Section}{Source}

SpatialPolygonDataFrame objects are downloaded from GIS shapefiles provided
by the California Department of Pesticide Regulation:
\url{http://www.cdpr.ca.gov/docs/emon/grndwtr/gis_shapefiles.htm}
\end{Section}
%
\begin{Section}{Note}

If this function returns an error (because the FTP site is down, for
example), check your working directory. You may want to change it back from
a temporary directory.
\end{Section}
%
\begin{Examples}
\begin{ExampleCode}
## Not run:
trinity_shp <- pull_spdf("trinity", download_progress = TRUE)
plot(trinity_shp)

del_norte_shp <- pull_spdf("08", "township", download_progress = TRUE)
plot(del_norte_shp)

## End(Not run)
\end{ExampleCode}
\end{Examples}
\inputencoding{utf8}
\HeaderA{purexposure}{purexposure: A package for working with CA Pesticide Use Registry data.}{purexposure}
\aliasA{purexposure-package}{purexposure}{purexposure.Rdash.package}
%
\begin{Description}\relax
The \code{purexposure} package provides functions to pull data from California's
Pesticide Use Registry (PUR), as well as to calculate exposure to and
visualize active ingredients present in applied pesticides. The main function
categories are \code{find\_*}, \code{pull\_*}, \code{calculate\_*}, and
\code{plot\_*}. These are the most important functions from each category:
\end{Description}
%
\begin{Section}{find\_* functions}

\code{find\_} functions help with searches of PUR chemical, county, and
product codes.
\begin{itemize}

\item \code{find\_chemical\_codes}: Pull active ingredient chemical codes
from PUR Chemical Lookup Tables.
\item \code{find\_counties}: Find California county codes or names.
\item \code{find\_product\_name}: Find Pesticide Product names and registration
numbers from PUR Product Lookup Tables.

\end{itemize}

\end{Section}
%
\begin{Section}{pull\_* functions}

\code{pull\_} functions facilitate downloading data from the CA Department
of Pesticide Regulation's website.
\begin{itemize}

\item \code{pull\_raw\_pur}: Pull raw PUR data by counties and years.
\item \code{pull\_clean\_pur}: Pull cleaned PUR data by counties, years, and
active ingredients.

\end{itemize}

\end{Section}
%
\begin{Section}{calculate\_* function}

The \code{calculate\_exposure} function calculates exposure (in kg/m\textasciicircum{}2) to
applied pesticides for a given location, buffer extending from that location,
time period, and active ingredients.
\end{Section}
%
\begin{Section}{plot\_* functions}

\code{plot\_} functions help with visualizations of application.
\begin{itemize}

\item \code{plot\_county\_application}: Plot pesticide application by county,
summed by section or township.
\item \code{plot\_exposure}: Plot exposure to applied pesticides at a
particular location.
\item \code{plot\_application\_timeseries}: Plot time series of active
ingredients in applied pesticides.

\end{itemize}

\end{Section}
\inputencoding{utf8}
\HeaderA{scale\_fill\_gradientn2}{Include alpha in ggplot2::scale\_fill\_gradientn().}{scale.Rul.fill.Rul.gradientn2}
%
\begin{Description}\relax
This function adds an `alpha` argument to scale\_fill\_gradientn() from the
ggplot2 package.
\end{Description}
%
\begin{Usage}
\begin{verbatim}
scale_fill_gradientn2(..., colours, values = NULL, space = "Lab",
  na.value = "grey50", guide = "colourbar", colors, alpha = NULL)
\end{verbatim}
\end{Usage}
\inputencoding{utf8}
\HeaderA{spdf\_to\_df}{Convert county SpatialPolygonsDataFrame to a tidy data frame.}{spdf.Rul.to.Rul.df}
%
\begin{Description}\relax
\code{spdf\_to\_df} converts a SpatialPolygonsDataFrame object returned from
the \code{pull\_spdf} function to a data frame.
\end{Description}
%
\begin{Usage}
\begin{verbatim}
spdf_to_df(spdf)
\end{verbatim}
\end{Usage}
%
\begin{Arguments}
\begin{ldescription}
\item[\code{spdf}] A SpatialPolygonsDataFrame object returned from
the \code{pull\_spdf} function.
\end{ldescription}
\end{Arguments}
%
\begin{Value}
A data frame with 24 columns if the \code{spdf} object is on the
section level and 23 columns if the \code{spdf} object is on the township
level.
\end{Value}
%
\begin{Examples}
\begin{ExampleCode}
## Not run:
df <- spdf_to_df(pull_spdf("fresno"))
df2 <- spdf_to_df(pull_spdf("sonoma"))

# use df_plot() function to easily plot the output data frames:
df_plot(df)
df_plot(df2)

## End(Not run)
\end{ExampleCode}
\end{Examples}
\inputencoding{utf8}
\HeaderA{tibble\_to\_vector}{Return a character vector from a tibble with one column.}{tibble.Rul.to.Rul.vector}
%
\begin{Description}\relax
\code{tibble\_to\_vector} takes a tibble with one column and returns the
values in that column as a character vector.
\end{Description}
%
\begin{Usage}
\begin{verbatim}
tibble_to_vector(tib)
\end{verbatim}
\end{Usage}
%
\begin{Arguments}
\begin{ldescription}
\item[\code{tib}] A tibble with only one column.
\end{ldescription}
\end{Arguments}
%
\begin{Details}\relax
This is a helper function for \code{pull\_raw\_pur}, \code{pull\_clean\_pur},
and \code{pur\_filt\_df}.
\end{Details}
%
\begin{Value}
A character vector.
\end{Value}
\printindex{}
\end{document}
